\documentclass[11pt]{article}
\usepackage[letterpaper, margin=1in]{geometry} %package that allows changes in margins and header/footers
\newcommand{\res}[1]{\noindent \textcolor{blue}{{#1}} \\}
\newcommand{\jri}[1]{\noindent \textcolor{red}{{#1}} \\}
\newcommand{\lev}[1]{\noindent \textcolor{green}{{#1}} \\}
\newcommand{\mbh}[1]{\noindent \textcolor{Dandelion}{{#1}}\\}
\usepackage{lipsum}
\usepackage{url}
\usepackage[usenames,dvipsnames]{color}
\usepackage{multirow}
\renewcommand{\baselinestretch}{1.5}
\usepackage{natbib} % package for formatting citations
\bibliographystyle{AJB} %indicating bibliography uses the AJB.bst file for 

\begin{document}

\noindent Dear Dr. Schnabel and additional editorial staff at the \emph{American Journal of Botany}, \\

We would like to thank you and two anonymous reviewers for your helpful comments and suggestions regarding our manuscript.  In the attached revision, we have addressed the majority of issues raised, but, while we feel that suggested new experiments would provide interesting results, they would require considerable extra time (many months) and are outside of the scope of this current work.  In addition to our revision, we include a detailed response to reviewers below. We hope that you are willing to consider our revised manuscript as a resubmission.\\\

\noindent Sincerely,\\

\noindent Jeffrey Ross-Ibarra and Matthew Hufford

\section*{Response to Associate Editor:}

I thought that the neighbor-joining analysis was of little value, because it was not clear to me just how the result could be expected to provide information about Hopscotch genotype.  Why would you expect individuals with similar Hopscotch genotypes to cluster together?  Also, although you don't provide any information on the distribution of variable sites among accessions and taxa, your data set of only 48 and 40 segregating sites is almost certainly not sufficiently variable to resolve relationships among the very large number of accessions you were analyzing.  As a result, you produce mostly a huge polytomy, with a small amount of structure that appears to reflect the genetic similarities and differences already evident in Table 1.  The trees themselves are difficult for a reader to interpret, because none of the accession names at the branch tips (other than TIL) are defined in the text.\\

\res{We agree entirely with this assessment of the Neighbor Joining trees. 
Our intention with the NJ trees was to show that there is no clear signal of selection on the \emph{Hopscotch} genotype across sequence in the \emph{tb1} genomic region.
Presumably, if the \emph{Hopscotch} element were under strong selection within teosinte populations we would expect to see \emph{tb1} haplotype groups in an NJ tree due to linkage disequilibrium and Hill-Robertson effects.
In fact, the original discovery of selection on the \emph{Hopscotch} insertion during maize domestication was based on such a signal from linked sequence 60kb distant from the \emph{Hopscotch}. \mbh{CITE} 
We agree with the reviewer that the trees could be confusing and difficult to interpret, and the large polytomies do not provide information beyond our assertion that there is not strong haplotype structure broadly in the \emph{tb1} genomic region that correlates with the \emph{Hopscotch} genotype.
We have therefore opted to keep the NJ trees in the supplement where interested readers can view them but where they don't unnecessarily complicate our main story.
In addition, we have added explanation of the accession names.}

\jri{need to add explanation of accession names in separate table or in figure legend.} \mbh{I'd add this to the legend}

Phenotyping experiments - These are curious experiments.  First, why did you not generate a population of individuals with known genotypes and then test for the effect of genotype on phenotype?  Why take a random sample and just hope that you will get the necessary number of individuals of each genotype?  How can you randomize the growing conditions with respect to genotype, if you don't know genotypes ahead of time but instead determine them after all the growing measurements are taken?\\

\res{We agree that a planned grow-out with previously determined genotypes could be useful.  
However, prior work has already shown that \emph{Hopscotch} has a phenotypic effect in isogenic and inbred backgrounds, \mbh{CITE} and our interests lie in the effects of the \emph{Hopscotch} in genetic backgrounds occurring within nature.
Moreover, our sampling of individuals from this population for our grow out was not random.  
We initially genotyped a single individual from each sampling site within the population and then targeted our sample for the grow out to include individuals from sampling sites where individuals were homozygous or heterozygous for the \emph{Hopscotch} allele.
Our reasoning in using this approach was that we wanted a high proportion of both \emph{Hopscotch} positive and \emph{Hopscotch} negative individuals for phenotypic comparisons and, given that the \emph{Hopscotch} allele is typically rare in teosinte, we wanted to enrich our sample for this allele. 
We were overly successful in this approach and had a higher frequency of individuals with the \emph{Hopscotch} positive allele.
For the Phenotyping 2 experiment we therefore selected individuals from a mixture of \emph{Hopscotch} positive and negative sites.
Despite this modified sampling approach, we still ended up with a higher frequency of \emph{Hopscotch} positive individuals, perhaps due to the fact that this allele appears to be under positive directional selection within the San Lorenzo population.
We have clarified this sampling methodology on page \mbh{X}, lines \mbh{XX}.
Finally, plants were situated randomly in the greenhouse prior to genotyping.
Regardless of the fact that genotypes were eventually determined, this still represents a randomized experimental design.}

Why are sample sizes not reported?  You provide total samples sizes (eg, 206 or 216 for Phenotyping 1, depending on whether you read the Methods or the Results), but not sample sizes for each genotypic class.\\

\res{We have reported sample size information for all three genotypes on page \mbh{X} lines \mbh{XX} (Phenotyping 1) and on page  \mbh{X} lines \mbh{XX} (Phenotyping 2).  
We have also corrected our typo for sample size in Phenotyping 1 and the Results and Methods sections are now consistent.} \jri{I didn't see sample sizes for each of the genotpyes. are they there? if not, please add.}

It was not clear to me what ``to detect the observed effect" means in lines 12-13 of p. 9.  Your Phenotyping 1 results show no effect of genotype on tillering.  Thus, it is not clear what effect you are referring to.\\

What was the purpose of the Phenotyping 2  experiment?  You already showed no effect of genotype on phenotype using populations where you are most likely to see segregation for the insertion, so it was not clear to me why sampling more broadly, especially from populations with no evidence of the Hopscotch insertion, would be expected to improve your understanding of this relationship.\\

\res{On Day 40 of the Phenotyping 1 experiment we saw a weak positive correlation between the \emph{Hopscotch} insertion and tillering index (p=0.0848).  
Using data from Phenotyping 1, we conducted a power analysis to determine the number of individuals we would need in both homozygous genotypic classes to observe a significant result.
This analysis indicated we would need 71 individuals in both classes to detect a significant difference.
We have clarified this on page \mbh{X}, lines \mbh{X}.}

\mbh{I don't think the next paragraph is necessary}

\res{Note that all the sampling for both phenotyping experiments was limited to a single large population from which we had a large number of seed collected. Seed was collected from various "sampling sites" within the contiguous population at which we had estimated Hopscotch allele frequency. For our more broad genotyping of both landrace maize, ssp. \emph{parviglumis}, and ssp. \emph{mexicana} we sampled as many populations as we could. For most of these populations the prevalence of the Hopscotch was previously unknown, and we were interested in having an idea of the overall distribution and frequency of the Hopscotch throughout these locations.}

The discussion of introgression, genetic drift, and selection in your Discussion section seems to lack coherence.  It sometimes focuses on explaining the unexpectedly high frequency of Hopscotch in present-day populations, whereas other times it seems to be addressing the presence or absence of Hopscotch in teosinte in general, and it also sometimes seems to be addressing selection on the tb1 locus. 

\res{We have edited this section for clarity to better emphasize our main point that the \emph{Hopscotch} allele, which is known to produce a domesticated phenotype of reduced tiller number, is at higher than expected frequency in a subset of teosinte populations and may play an adaptive role in certain environments encountered by teosinte.  
Surprisingly, tiller number is not a major phenotypic difference between teosinte plants with and without the \emph{Hopscotch} insertion, perhaps indicating the insertion affects multiple phenotypes in teosinte}

\section*{Response to Reviewer 1}

Reviewer \#1: In their paper `Natural variation in teosinte at the domestication locus teosinte branched1 (tb1)', Vann et al survey a large sample of maize and teosinte individuals for the presence of the Hopscotch transposable element upstream of tb1. The authors find that Hopscotch is more widespread, across a large sampled area in Mexico, than previously thought, particularly in parviglumis. The authors go on to sequence regions up and downstream of Hopscotch in subset of their initial sample. In analyzing their sequence data, they find no evidence for recent introgression of the maize Hopscotch locus into wild teosintes. They also find evidence for selection acting on tb1 in parviglumis. In greenhouse experiments, no difference in tiller index or tiller number was observed between teosinte lines with or without Hopscotch. 

They present an interesting result, their conclusions are supported by their data, and the methods they use are appropriate. However, I feel like they need to make some further effort to understand what is happening mechanistically. What is Hopscotch doing in maize vs these sampled teosintes? I feel like the expression analyses and phenotyping experiments they suggest should be part of a later study should be part of this study. At a minimum, qRT-PCR experiments should be performed assessing tb1 expression in their with/without Hopscotch lines. Expression analyses of gt1, te1, and tru1 would be great, since they invoke variation at these loci to explain their results.

\res{We attempted to measure expression of \emph{tb1} following the protocol of \citep{hubbard2002expression} but were unable to get reasonable results. 
Subsequent discussion with colleagues revealed teosinte expression of \emph{tb1} is very difficult to detect except in early ear tissue. Extensive expression analysis of \emph{tb1} and a number of other genes is a good suggestion for follow-up work but would be a substantial undertaking and indeed more appropriate for another paper. }

Another potential avenue, that the authors also suggest, lies in more directed and exhaustive greenhouse experiments. Variation in red and far-red light has been shown to be important in regulating both tb1 (in Sorghum) and gt1 in maize. Growth chamber experiments (with or without supplemental far red light, with or without Hopscotch), coupled to expression analyses in seedlings, seem like they should be possible.  In the greenhouse experiments that were done, phenotyping was incomplete. Apart from tillering, Hopscotch and natural variation at the tb1 locus appears to affect ear architecture and internode length. Was there any evidence for variation in these other traits? In the longer term, it would be nice to see what happens when you introgress these particular parviglumis loci into maize.\\ 

\res{We did not measure ear architecture or internode length. 
While \emph{tb1} does impact these traits, our primary interest was in the effect of \emph{tb1} on tillering as a potentially ecologically relevant phenotype.}

\res{Both additional greenhouse and expression analyses would indeed be useful avenues of further investigation.
Nonetheless, these are extensive additional experiments which we are unable to include at this time. 
We have, however, added text to the conclusion (\mbh{Page X, Lines X}) of this paper discussing their utility and the need for further work to elucidate the role of \emph{tb1} in branching architecture in teosinte:
``Future studies should examine expression levels of \emph{tb1} in teosinte with and without the \emph{Hopscotch} insertion and further characterize the effects of additional loci involved in branching architecture (e.g. \emph{gt1}, \emph{tru1}, and \emph{te1}) as well as include a more exhaustive phenotyping including all traits.  
These data, in conjunction with more exhaustive phenotyping, should help reveal the ecological significance of the domesticated \emph{tb1} allele in natural populations of teosinte."}

One minor issue: in Fig. 1, it would be nice to see where the Balsas River Basin is in relation to the sampling shown. 

\res{We have added the Balsas River Basin in a revised version of Figure 1.} \jri{LV please add, don't wait for Matt.} \mbh{see the R tutorial I sent, LV}


\section*{Response to Reviewer 2} 
Although this paper brings up an interesting question regarding the role of tb1 standing variation in teosinte, it is not able to answer it completely.  It presents results that have been published previously about the presence of hopscotch in teosinte, maize, and mexicana. 

\res{Reviewer 2 is correct in stating that it was known that the Hopscotch was present in some teosinte individuals; however, previous studies sampled a small fraction of the populations and individuals per population that we present here and thus did not provide comprehensive estimates. 
Our study is not only concerned with presence or absence of the \emph{Hopscotch} allele in these populations but also the frequency at which it segregates.  
This finer level of detail sheds light on the evolutionary history of the \emph{Hopscotch} insertion in teosinte. 
We have added clarifying text to the introduction: ``The effects of the \emph{Hopscotch} insertion have been studied in maize (Studer \emph{et al.} 2011), and analysis of teosinte alleles at \emph{tb1} has identified functionally distinct allelic classes (Studer and Doebley 2012), but little is known about the role of \emph{tb1} or the \emph{Hopscotch} insertion in natural populations of teosinte. Previous studies have confirmed the presence of the \emph{Hopscotch} in samples of ssp. \emph{parviglumis}, ssp. \emph{mexicana}, and landrace maize; yet little is known about the frequency at which the \emph{Hopscotch} is segregating in natural populations."}

Nevertheless, the manuscript is worth publishing as it does present data that may help rule out some possible explanations as to why tb1 is so common in parviglumus.\\

\res{Thank you; we hope that our paper in conjunction with future research and experiments can help to further the understanding of ecological significance of both \emph{tb1} and the \emph{Hopscotch} insertion in teosinte.}

This paper is fairly maize-centric. A more general message is included in the discussion, not as much in the introduction, and not at all in the abstract. This could be easily remedied.\\

\jri{tweak intro and abstract, add a bit to discussion maybe, explain text changes here.}
\lev{Jeff and Matt, this is already in the intro. I don't see what more we can say that would be appropriate. ``The \emph{tb1} locus appears to play an important role in the shade avoidance pathway in \emph{Zea mays} and other grasses and may therefore be crucial to the ecology of teosinte."}
\jri{add some discussion about domesitcaiton traits in other species, weedy rice, etc. }

Methods and statistics were appropriately used as far as my expertise allows me to judge.
Specific suggestions follow:

Introduction:
Page 4, Lines 7 - 9: You say "Additionally, many domesticates show reduced genetic diversity when compared to their wild progenitors, and an understanding of the distribution of diversity in the wild and its phenotypic effects has become increasingly useful to crop improvement (Kovach and McCouch, 2008).".  Can you tell us why an understanding is useful? Can you give some examples?\\

\res{We have expanded on this concept in the introduction using \emph{Oryza sativa} as an example: ``For example, \emph{Oryza rufipogon}, the wild progenitor of domesticated rice, has proven useful for the integration of a number of beneficial QTL controlling traits such as grain size and flowering time into domesticated rice (Kovach and McCouch 2008)."} \jri{read matt's teosinte TIG paper, there are good cites there about use of teosinte specifically.}

8) Page 5, line 12 - 14: You say "The tb1 locus appears to play an important role in the shade avoidance pathway in Zea mays and other grasses and may therefore be crucial to the ecology of teosinte (Kebrom and Brutnell, 2007; Lukens and Doebley, 1999)." Please clarify; do the references belong to the first half of the sentence, supporting tb1 playing a role in Zea mays, or the second half of the sentence, supporting tb1 possibly being crucial to the ecology of teosinte? Which of the two distinct things have actually been studied? If the first, could you tell us how tb1 has been shown to work in maize under shade conditions? If the second, was it actually studied, or merely hypothesized in teosinte?\\ 

\res{We have clarified issues regarding the references for this section of the paper.}

9) Page 5, Lines 15 - 16: you say you ``aim to characterize the distribution of the Hopscotch insertion in parviglumis, mexicana, and landrace maize, and to examine the phenotypic effects of the insertion in parviglumis."  The mexicana seem a bit tacked on and not much discussed. Did you phenotype mexicana? Can you present the data here, if so? Does the analysis of this group of teosintes add anything to your conclusions?\\

\res{Reviewer 2 is correct in noting that ssp. \emph{mexicana} is not as prevalent in our paper as ssp. \emph{parviglumis}. This fact is due to limitations in access to accessions of ssp. \emph{mexicana} as field collections for ssp. \emph{mexicana} are not nearly as extensive as those of ssp. \emph{parviglumis}. Additionally ssp. \emph{mexicana} is more difficult to cultivate in a greenhouse setting. We felt that it would be important to include ssp. \emph{mexicana} in our study, especially since it is known that gene flow occurs between ssp. \emph{mexicana} and landrace maize. We have noted in our methods that we were only able to phenotype a population from which we had a large number of seed collections.} \jri{just say we don't have as much seed from mexicana, which is true. it's just as easy to grow in the greenhouse. also I'd think of adding a sentence to the discussion mentioning mexicana. if it is in mexicana too, doesn't that suggest it's pretty old, given the divergence time between mexicana and parviglumis? or do we think it's in mexicana because of gene flow with parv or maize? this could be worth speculating on.}
\lev{Hmm. Well didn't Matt's paper on introgression support a significant lack of gene flow from maize to mex around domestication loci? So how likely would it be that it is in mex because of gene flow? So would it make more sense to argue that it's there because it is old and hasn't been lost? Would it make sense too because it was estimated to predate the domestication of maize? Maybe it is in parv at a higher frequency because it's 'doing' something, even though we didn't pick up an effect on tillering, and maybe it's just drifting in maize, which could explain why it is at a much lower frequency in mex?}

10) Materials and Methods:
Page 5, lines 22 - 23: how many individuals per accession? A fixed number, or variable, and depending on what? This can be mentioned so the reader does not have to go to the appendix or supplements.\\ 

\res{We have added a range of values (1-38 individuals) per population in the methods: ``We sampled 1,110 individuals from 350 accessions (247 maize landraces, 17 \emph{mexicana} populations, and 86 \emph{parviglumis} populations; ranging from 1-38 individuals per population) and assessed the presence or absence of the \emph{Hopscotch} insertion (Appendix 1 and Appendix 2, See Supplemental Materials with the online version of this article)." \jri{I'd also add a median or mean, or note how many populations had >8 individuals or somet other way for them to get a sense of the data as a range by itself is hard to think about}}

%JRI STOPPED HERE

11) Page 5, Line 28: mention how big the hopscotch element is.  Can it be amplified in one amplicon? Again, this would be useful to know without having to look in appendix or supplements.\\

\res{The size of the amplicon is listed in now clarified in the Materials and Methods following the primer sequences: ``Homozygotes show a single band for absence of the element ($\sim$300bp) and two bands for presence of the element ($\sim$5kb, amplification of the entire element, and $\sim$1.1kb, amplification of part of the element), whereas heterozygotes are three-banded (Appendix 2, See Supplemental Materials with the online version of this article)."}

12) Page 6 line 6: "When only one PCR resolved well, we scored one allele for the individual".  Sorry, what does this mean? That you have one band (one assumes the short one?). If you are talking presence/absence, how do you know the allele?\\ 

\res{We have clarified in the text. The 2 PCRs are for the two different alleles. If only one PCR resolved we only included one allele for that individual instead of trying to impute a diploid genotype. We have included the following sentence in the text: ``Since we had a PCR for each allele, if only one PCR resolved well, we scored one allele for the individual."}

13) Page 7 lines 14 - 15 "These analyses only included populations in which eight or more individuals were sampled."  How many populations did that include?\\
 
\res{We included 32 populations for these analyses, and have edited the text to reflect this number: ``These analyses only included populations in which eight or more individuals were sampled, totaling 32 populations."}

14) Page 7 lines 17 - 26: how many year's worth of environmental data went into your analysis? Can you hypothesize that tb1 differences can be correlated with current climactic data if the changes and subsequent selections must have happened many generations ago?\\
 
\res{The environmental data used from Pyh\"aj\"arvi  \emph{et al.} (2013) was one year's worth of environmental data. A previous study by Hufford \emph{et al.} (2012) has shown that environmental conditions in areas where teosinte grows have not changed significantly over the last 10,000 years. We have clarified the use of one year's worth of data in the text: ``Environmental data (One year's worth) were obtained from \url{www.worldclim.org}, the Harmonized World Soil Database (FAOHWSD) and \url{www.harvestchoice.org} and summarized by principle component analysis following Pyh\"aj\"arvi  \emph{et al.} (2013)."}

15) Page 7 lines 22 - 23: "We used genotyping and covariance data from Pyhajarvi et al. (2013) for BayEnv" what kind of dataset is this? Teosinte?\\ 

\res{We have clarified the dataset in the paper: ``We used teosinte (ssp. \emph{parviglumis} and ssp. \emph{mexicana} genotyping and covariance data from Pyh{\"a}j{\"a}rvi \emph{et al.} (2013) for BayEnv, with the \emph{Hopscotch} insertion coded as an additional SNP. SNP data from Pyh{\"a}j{\"a}rvi \emph{et al.} (2013) were obtained using the MaizeSNP50 BeadChip and Infinium® HD Assay (Illumina, San Diego, CA, USA)."}

16) Page 8 lines 20 - 21:" We created haplotype blocks using a custom Perl script that grouped SNPs separated by less than 5kb into haplotypes." Does LD data in your parviglumus support 5Kb?\\ 

\res{hmm. Matt why did you use 5kb when you did this for your study?}
\jri{read Matt's paper, explain to reviewer}

17) Page 9 lines 2 - 3: "Plants were watered three times a day by hand and with an automatic drip containing 10-20-10 fertilizer." This sounds odd; by hand or automatically? Or you turned on the drip by hand? In which case, it was not automatic… \\

\res{We have edited the text to clarify: ``Plants were watered three times a day with an automatic drip containing 10-20-10 fertilizer, which was supplemented with hand watering on extremely hot and dry days."}

18) Page 9 lines 7 - 8: "Culm diameter is not believed to be correlated with tillering index or variation at tb1. F" So, why did you do it? Because of the last sentence in the materials and methods? Perhaps do not mention here that is not correlated; wait until the end of the materials and methods where you seem to explain it.\\ 

\res{Thank you for the suggestion, we have edited it in the text and added it to the end of the Materials and Methods section: ``Additionally, in order to compare any association between \emph{Hopscotch} genotype and tillering and associations at other presumably unrelated traits, we performed an ANOVA between culm diameter and genotype using the same general linear model in SAS. Culm diameter is not believed to be correlated with tillering index or variation at \emph{tb1} and is used as our independent trait for phenotyping analyses."}

19) Page 9 lines 9 - 10: "We genotyped individuals for the Hopscotch insertion following the protocols listed above." The PCR protocols or the sequencing protocols?\\ 

\res{We have clarified this in the manuscript.}

20) Results:
Page 10 lines 5 - 6: "Within our parviglumis and mexicana samples we found the Hopscotch insertion segregating in 37 and four populations, respectively"; remind us how many parviglumis and Mexicana samples you had total? So we can see if this comes out to a very different ratio than in maize.\\ 

\res{We have added ratios to the text: ``Within our \emph{parviglumis} and \emph{mexicana} samples we found the \emph{Hopscotch} insertion segregating in 37 (37 out of 86) and four (4 out of 17) populations, respectively, and at highest frequency in the states of Jalisco, Colima, and Michoac\'{a}n in central-western Mexico.}"

21) Page 10 lines 7 - 9: "Using our Hopscotch genotyping, we calculated differentiation between populations (FST) and subspecies (FCT) for populations in which we sampled eight or more alleles." What do you mean, 8 or more alleles? I thought you only had presence or absence; how are alleles differentiated with your PCR?\\ 

\res{We believe we addressed this with editing for your previous comment concerning uncertainty of the PCRs we were using. As edited into the Materials and Methods section, there is a PCR that produces a specific size for each allele. We have edited subsequent text accordingly.}

22) Page 13 lines 25 - 27: "and genetic data support ongoing gene flow between domesticated maize and both mexicana and parviglumis in a number of sympatric populations (Hufford et al., 2013; Ellstrand et al., 2007; van Heerwaarden et al., 2011). Add Warburton, et al., 2011.  Gene flow among different teosinte taxa and into the domesticated maize gene pool.  Genet Resour Crop Evol 58:1243-1261.\\ 

\res{We have added this reference as suggested.}

23) Page 14 line 15 to page 15 line 14: too long a paragraph. Break into two, perhaps at page 14 line 32.\\ 

\res{We have split the paragraph into two.}

\bibliography{refs}

\end{document}