\documentclass[11pt]{article}
\usepackage[letterpaper, margin=1in]{geometry} %package that allows changes in margins and header/footers
\newcommand{\jri}[1]{\noindent \textcolor{red}{{#1}} \\}
\newcommand{\lev}[1]{\noindent \textcolor{magenta}{{#1}} \\}
\newcommand{\mbh}[1]{\noindent \textcolor{Dandelion}{{#1}}\\}
\usepackage{lipsum}
\usepackage{url}
\usepackage[usenames,dvipsnames]{color}
\usepackage{multirow}
\renewcommand{\baselinestretch}{1.5}

\begin{document}

Associate Editor and Reviewer comments:

Dear Dr. Hufford and coauthors - 
    Thank you for your thorough response to previous criticisms of your manuscript and your attention to improving the manuscript.  I sent the current manuscript to one of the previous reviewers (Reviewer 2 here) and to a new reviewer (Reviewer 1 here) to get a more independent view of the work.  Reviewer 2 has recommended acceptance of the manuscript, albeit a bit begrudgingly, whereas Reviewer 1 recommends rejection and voices some of the same objections that arose in the original rounds of review - the lack of data on tb1 expression in the presence of Hopscotch and the confusion over explanations related to Hopscotch and explanations about evolution of the tb1 gene as a whole (eg, see Reviewer 2’s statements, “the paper is written explicitly assuming a connection between Hopscotch and tb1 expression in teosinte, and often conflates variation at the tb1 locus with presence or absence of Hopscotch” and “veering in focus between the purported object of this study, hopscotch
elements, and the tb1 gene itself”).  Your Abstract provides an example of how a reader could be confused about the focus of the paper: “Population genetic signatures are consistent with selection on this locus revealing a potential ecological role for Hopscotch in teosinte, but a greenhouse experiment does not detect a strong association between tb1 and tillering in teosinte.”  From this sentence, the reader would infer that you tested for selection on Hopscotch and did the greenhouse experiment on different tb1 genotypes, when the reverse is actually true.
     Because these reviewers reach such different opinions, I have spent a considerable amount of time giving the manuscript a very careful reading and also reviewing some of the most important literature you cite.  Based on this review, I’ve concluded that much of the criticism from Reviewer 1 arises because you have not clearly stated your hypotheses.  You voice some unknowns, you make a couple of hypotheses, and you have some aims, but none of this is organized in a way that gives the reader a clear picture of the logic of your thinking or the structure of the study.  For example, you have a hypothesis in your final paragraph of the Introduction (about density effects on tillering), but it is not actually something you tested, because you did not test different densities.  You would be well served, I think, by discussing amongst yourselves exactly what hypotheses the various pieces of your study were designed to test and then stating those clearly at the end of the
Introduction.  If you state those hypotheses clearly, then you are much less likely to be accused of mix-and-match arguments that switch between Hopscotch itself and the tb1 locus as a whole.  Based on my understanding of your paper and some of the major literature you cite, you seem to be testing hypotheses about (1) the frequency of the Hopscotch insertion in natural population (genotyping survey) based on some preliminary studies; (2) the effect of Hopscotch on tillering (greenhouse experiment) based on what is known about the effect of Hopscotch in maize and the evidence that Hopscotch was present in teosinte populations prior to domestication; (3) the possibility of explaining the high frequency of Hopscotch in teosinte by introgression from maize (LD tests using sequences); and (4) the possibility of past or present selection on the entire locus in natural teosinte populations (Tajima D test).  Of these, the weakest part is the selection analysis, and I think you might
be better off dropping that part and just focusing on the first three hypotheses.  Once you have some of the expression data that all three reviewers have called for, you could return to the question of the ecological role for Hopscotch and tb1 in natural populations.
\lev{No on expression data and additional experiments. I thought our hypotheses/aims were rather clearly stated, but perhaps we can reword if you guys think that they need clarification.}
    In addition to these concerns, I found several oversights in the paper.  For example, you state that there are 1110 individuals sampled, but based on the numbers of populations and landraces you give and the average sample size of each, the total number of individuals should be 1627 (ave. 2 accessions/landrace x 247 landraces + ave. 11 ind/population x 103 teosinte populations).  The problem appears to lie, at least in part, in a confusion about alleles vs individuals, since most of the landraces shown in Appendix S2 show 2 alleles sampled, not 2 individuals.  Even if the number is actually 1110, then I found it troubling that nearly 25% of those could not be successfully genotyped.  Second, your figure showing the PCR assay (Appendix S4, not S2 as stated in the text) has an incorrect heterozygote genotype (assuming “no Hop/Pif” is the heterozygote) that does not show the three bands you describe in the text.  Third, you state that you calculate Watterson’s estimator of
population mutation rate, but you never mention it again after the Methods.
\lev{Will check numbers. I think the confusion is in the individuals vs. alleles.}
    In summary, despite the significant improvements to the current manuscript, I cannot recommend publication, due to the consistent call for expression data from three reviewers, continued confusion about the focus of the paper, and problems relating to sampling and genotyping.  I agree with Reviewer 1 that there are “excellent data in this manuscript that deserves to be published in some form”, and I hope you can find a way to revise the manuscript to be acceptable in a different venue.

Yours sincerely,
Andrew Schnabel
Associate Editor


Reviewer #1: Vann review AJB 2014

Summary
This manuscript is a study of the phenotypic effect in a wide sample of teosinte populations of the Hopscotch retrotransposon known to regulate expression of teosinte branched1 (tb1) in maize.  It also contributes to our understanding of the population genetic history of teosinte and of the Hopscotch element.  The basic argument of the manuscript is as follows: 1) Hopscotch regulation of tb1 has been shown to be important in controlling plant architecture in domesticated maize, 2) Hopscotch segregates in natural populations of teosinte (ancestor of maize), therefore 3) what is the  effect of Hopscotch presence or absence on tb1 expression and phenotype in teosinte.  The authors focus on tillering and do not explore other known effects of tb1 on internode elongation or sexuality.  

The main problem for me about this argument is that there is an implicit link between the presence or absence of Hopscotch and levels of tb1 expression in teosinte plants, and the effect this has on tillering.  There is excellent data in this manuscript that deserves to be published in some form, but I do not believe that the argument that is constructed is valid without showing that Hopscotch actually affects tb1 expression in teosinte. It would be a different matter if the paper were merely focused on the effect of the presence or absence of Hopscotch, but the paper is written explicitly assuming a connection between Hopscotch and tb1 expression in teosinte, and often conflates variation at the tb1 locus with presence or absence of Hopscotch.  This is a problem, especially when the data show little effect of Hopscotch presence or absence on the measured phenotypes.  Below I expand on this summary and also present some other queries.  
\lev{no expression analyses. should we downplay any mention of expression in the paper? or somehow make it more clear that it will be a nice follow up in another study? If all the reviewers we have had so far feel expression is crucial, I do think that we'll have to find a way to address the lack of expression in the paper}
Introduction
page 5, lines 21-28.  The logic here is unclear.  The Lukens and Doebley experiment showed that maize lines with an introgressed teosinte tb1 allele were affected by shading.  It does not follow that the domesticated allele (Hopscotch containing allele) should play a role in teosinte ecology, especially since maize with the Hopscotch allele is insensitive to shading.  This argument would make more sense if the authors had shown that Hopscotch alleles affect tb1 expression in teosinte. \lev{I don't understand. If Hopscotch = taller plant with fewer tillers then why couldn't it play a role in the ecology of teosinte? Our phenotyping obviously doesn't support this pattern in teosinte, but I don't understand how the hypothesis is a problem.}

Page 5, lines 28-30:  Absolutely nothing wrong with this aim, as far as it goes.  However, the paper tries to do more than just this, which is a problem because of the flaws in the argument summarized above.
["In this study we aim to characterize the distribution of the Hopscotch insertion in parviglumis, mexicana, and landrace maize, and to examine the phenotypic effects of the insertion in parviglumis."]

\lev{I don't think there is a big problem with this. I think it's clearly stated. Is the problem he is pointing out that we have explicit hypotheses about the phenotypic effects and that it may play a role in the ecology that we don't include in this statement?}
Materials and Methods
Page 6, line 30: Minor point, and I am not an expert on population genetics, but why say that you only include populations in which 8 or more chromosomes were sampled, rather than 8 or more individuals.  If I am missing something, it could be that others will too, so please explain.
\lev{This was just our sampling cut-off for analyses...}

Page 6, line 30- page 7, line 2: Testing the adaptive hypothesis.  I understand that in the end there was no evidence for Hopscotch allele frequency being correlated with environmental variables.  Was there evidence for any of the other SNPs being so correlated?  How robust is this analysis against false positives?  The adaptive hypothesis is attractive, but it is unclear why you would make it.  You do not make the claim in these lines, but it seems to me that it is the three classes of teosinte alleles that might be more likely to correlate with environment (since Studer et al. 2012 showed that they correlated with taxonomy) than the presence or absence of the Hopscotch element.  
\lev{I only used the Hopscotch, coded as a SNP. Tanja's paper would have included the other SNPs in the panel. I don't remember off the top of my head if she found any that were significant on Chromosome 1, but that is easily checked. Should we mention SNPs she identified in her paper her? Yes, the three classes of teosinte likely correlate with environment, but that's not something we address here. Should we take out BayEnv stuff all together? I like it in, but it doesn't add much I guess.}

\lev{\bf{DONE}}Page 9, line 15: In the STRUCTURE analysis, why was K=2 chosen?  Had you already done exploratory runs with a range of K values or is there some theoretical reason for assuming K=2.  Either way, please explain.
\lev{I think we chose K=2 because we expected things to group either into a teosinte cluster or a maize cluster}

\lev{\bf{DONE}} Page 10, line 12: Why is culm diameter not believed to correlated with tillering index?  Is there a reference for this? What do your results show?  More importantly, if you want to prove that Hopscotch has more effect on tillering than on other independent traits you really need to do more comprehensive phenotyping.  Please explain more fully what this particular analysis is meant to achieve.
["Culm diameter is not believed to be correlated with tillering index or variation at tb1 and is used as our independent trait for phenotyping analyses."]
\lev{This is cited as Briggs et al. 2007 in the text}\\
Discussion
Page 14, 3rd para: This paragraph starts off by examining explanations of genetic drift and natural selection for different frequencies of Hopscotch alleles amongst teosinte populations.  After discussing evidence for bottlenecks, the argument shifts to the lack of diversity in the 5' UTR of the tb1 gene, concluding that the results are consistent with the action of selection on the 5'UTR, and that there might be an ecological role for the gene in some populations.  This is a classic example of the veering in focus between the purported object of this study, hopscotch elements, and the tb1 gene itself.  In the end there is no explanation given for the hopscotch frequencies, instead the focus is switched back to the gene itself.  To the reader there is an implication (perhaps unintentional) that there is a causative link between Hopscotch and variation in the tb1 alleles, whereas I would say this has simply not been shown in teosinte, and the evidence in this manuscript would
argue against such a link.
\lev{I don't mind this paragraph. It ties in findings from previous studies that identified evidence of bottleneck in some of these pops with our findings, which i think is useful. We state that a population bottleneck could explain the pattern that we see in SLO, but that it cannot account for this pattern across multiple populations in the Jalisco region.}

Page 15, line 10: the argument at this point appears to be that we don't see the effect of Hopscotch in teosinte because of variation in other loci.  I would like more explanation of how that might occur, given that the Hopscotch element in maize upregulates tb1 expression so much that it explains most of the variation.  
\lev{There were other loci in the initial QTL paper that are involved in the difference in branching architecture. Plus we now know of a lot of other loci (mentioned in paper) that act in the SAS and could affect plant architecture.}

Page 15, line 14, discussion of Studer et al. (2012).  Studer et al. showed that there is an allelic series of tb1 in teosinte, and that this series is correlated with taxonomy of maize subspecies.  They also showed that differences between the three classes of tb1 alleles were significantly correlated with mean internode length of the uppermost lateral branch as well as two inflorescence traits, but that the classes did NOT correlate with variation in tillering.  Studer et al. concluded that there was evidence for allelic series only for the internode length and inflorescence characters, not for tillering.  Their explanation was NOT that tillering index was controlled by both tb1 and other loci, but that there was no evidence for variation in tb1 alleles affecting tillering.  Changes in other loci were invoked by Studer et al. (along with environmental effects on seed development) to explain why separate families differed from each other irrespective of the tb1 allele they
contained.  Again, at the end of this paragraph (page 15, lines 27-28), there is confusion over what is being explained, as the last sentence refers to a lack of correlation between Hopscotch genotype and tillering, but the discussion of the Studer et al. (2012) paper does not concern Hopscotch, but only an allelic series at the tb1 locus.  In fact, the Studer paper does not mention Hopscotch elements at all and the allelic series analyzed are only the coding region of the tb1 gene.
\lev{I think we should reread this section of the studer paper and check. If I remember correctly they found differences between their maize isogenic lines that were comparable to the differences they found among maize with introgressed segments of tb1}



Reviewer #2: This paper presents some truly tantalizing results: a higher frequency of Hopscotch  in teosinte than expected, some evidence for positive selection acting on tb1 in teosinte, and no evidence for recent introgression of Hopscotch into teosinte from maize. The methods used and statistical analyses are all appropriate and carefully conducted, and the authors don't overstate their conclusions. The paper, as it stands, is well written and easy to follow. I found this to be true before it had been revised, and the revisions have only improved it. 

However, it is extremely unlucky that the authors found no convincing connection between Hopscotch genotype and tillering, or between genotype and climatic variables. It is precisely because the results are so interesting that I was hoping for further functional characterization - either in the form of expression analyses (even of just tb1), or in the form of more complete phenotypic characterization. I do appreciate that this would require a significant input of time, at some risk of obtaining yet more negative data. Although the authors have already spent a lot of time and effort trying to detect an effect linked to Hopscotch, the ecological role of Hopscotch in teosinte (if any) remains elusive.

\lev{nothing to change for reviewer 2}

\end{document}